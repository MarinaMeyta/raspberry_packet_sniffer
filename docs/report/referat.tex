\newpage
\ESKDthisStyle{empty}
\paragraph{\hfill РЕФЕРАТ \hfill}
Курсовая работа содержит \ESKDtotal{page} страниц, \ESKDtotal{figure} рисунка, \ESKDtotal{bibitem} источников.

RASPBERRY PI, UBUNTU, WI-FI, TCPDUMP, PCAP, RASPBIAN, WIRESHARK, BASH, СНИФФЕР, СЕТЕВОЙ ТРАФИК.

Цель работы --- разработать и настроить систему пассивного перехвата пакетов по беспроводной сети Интернет (Wi-Fi) в научно-образовательных целях.

Проект выполнен с использованием следующих программных и аппаратных средств:
\begin{itemize}
  \item ОС Linux Ubuntu 15.04;
  \item ОС Linux Raspbian Jessie Light 4.1;
  \item Wi-Fi-адаптер TP-Link TL-WN722NC;
  \item tcpdump --- консольная утилита Unux для перехвата и анализа сетевого трафика;
  \item Wireshark --- программа-анализатор сетевого трафика с графическим пользовательским интерфейсом. 
\end{itemize}

Пояснительная записка выполнена согласно образовательному стандарту ВУЗа ОС ТУСУР 01-2013~\cite{os_tusur} при помощи системы компьютерной вёрстки \LaTeX.
