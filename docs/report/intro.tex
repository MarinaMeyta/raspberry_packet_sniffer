В настоящее время все больше устройств имеют возможность беспроводного выхода в сеть Интернет. Wi-Fi-соединение используется повсеместно за счет его удобства и развития беспроводных технологий. В связи с нарастающим применением беспроводных устройств внутри современных организаций необходимо построение системы безопасности не только корпоративных локальных сетей стандарта Ethernet, но также и беспроводных сетей стандарта Wi-Fi. 

Пакетные снифферы (устройства перехвата сетевых пакетов) могут быть использованы для анализа трафика в режиме реального времени либо же сравнения ожидаемого сетевого трафика и перехваченного. Данные устройства также могут быть использованы для сбора статистики, обнаружения сетвых атак или посторонних беспроводных устройств.

Пакетный сниффер на основе миникомпьютера Raspberry Pi и свободно распространяемого программного обеспечения может стать недорогой альтернативой подоброго рода устройств. 

В качестве задания на курсовую работу была поставлена задача разработать пассивный перехватчик пакетов на базе Raspberry Pi, Wi-Fi-адаптера и существующих программ для перехвата и анализа сетевого трафика.
