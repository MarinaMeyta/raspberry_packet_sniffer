\input{config}

\begin{document}

\newpage
\ESKDthisStyle{empty}

\begin{center}
 Министерство образования и науки Российской Федерации\\
 Федеральное государственное бюджетное образовательное учреждение высшего профессионального образования\\
 <<ТОМСКИЙ ГОСУДАРСТВЕННЫЙ УНИВЕРСИТЕТ СИСТЕМ УПРАВЛЕНИЯ И РАДИОЭЛЕКТРОНИКИ>> (ТУСУР)\\
 Кафедра комплексной информационной безопасности электронно-вычислительных систем (КИБЭВС)\\
\end{center}

\vfill

\begin{flushright}
\begin{minipage}{0.45\textwidth}
 \begin{flushleft}
  УТВЕРЖДАЮ\\
  заведующий каф. КИБЭВС
  \underline{\hspace{3cm}}А.А. Шелупанов \\
  <<\underline{\hspace{1cm}}>>\underline{\hspace{3cm}}2015г.\\
 \end{flushleft}
\end{minipage}
\end{flushright}

\vfill

\begin{center}
СОЗДАНИЕ ПАССИВНОГО ПЕРЕХВАТЧИКА ПАКЕТОВ ПО БЕСПРОВОДНОЙ СЕТИ???

Курсовая работа по дисциплине <<Безопасность сетей ЭВМ>>

Пояснительная записка к курсовой работе
\end{center}

\vfill
\begin{flushright}
\begin{minipage}{0.45\textwidth}
 \begin{flushleft}
  Выполнила: \\
  студентка гр. 722 \\
  \underline{\hspace{3cm}}М.В. Мейта \\
  <<\underline{\hspace{1cm}}>>\underline{\hspace{3cm}}2015г.\\
 \end{flushleft}
\end{minipage}
\end{flushright}

\vfill

\begin{flushright}
\begin{minipage}{0.45\textwidth}
 \begin{flushleft}
  Научный руководитель: \\
  аспирант каф. КИБЭВС \\
  \underline{\hspace{3cm}}А.К. Новохрестов \\
  <<\underline{\hspace{1cm}}>>\underline{\hspace{3cm}}2015г.\\
 \end{flushleft}
\end{minipage}
\end{flushright}

\vfill

\begin{center}
 Томск 2015
\end{center}

\newpage
\ESKDthisStyle{empty}
\paragraph{\hfill РЕФЕРАТ \hfill}
Курсовая работа содержит \ESKDtotal{page} страниц, \ESKDtotal{figure} рисунков, \ESKDtotal{table} таблиц, \ESKDtotal{bibitem} источника, \ESKDtotal{appendix} приложение.

RASPBERRY PI, UBUNTU, WI-FI, TCPDUMP, PCAP, RASPBIAN, WIRESHARK, BASH, СНИФФЕР, СЕТЕВОЙ ТРАФИК.

Цель работы --- разработать и настроить систему пассивного перехвата пакетов по беспроводной сети Интернет (Wi-Fi) в научно-образовательных целях.

Проект выполнен с использованием следующих программных и аппаратных средств:
\begin{itemize}
  \item ОС Linux Ubuntu 15.04;
  \item ОС Linux Raspbian Jessie Light 4.1;
  \item Wi-Fi-адаптер TP-Link TL-WN722NC;
  \item tcpdump --- консольная утилита Unux для перехвата и анализа сетевого трафика;
  \item Wireshark --- программа-анализатор сетевого трафика с графическим пользовательским интерфейсом. 
\end{itemize}

Пояснительная записка выполнена при помощи системы компьютерной вёрстки \LaTeX.


 % - содержание
\newpage
\ESKDstyle{plain}
\tableofcontents

\newpage
\ESKDstyle{plain}
\setcounter{section}{0}
\section*{Введение}
\addcontentsline{toc}{section}{Введение}
В настоящее время все больше устройств имеют возможность беспроводного выхода в сеть Интернет. Wi-Fi-соединение используется повсеместно за счет его удобства и развития беспроводных технологий. В связи с нарастающим применением беспроводных устройств внутри современных организаций необходимо построение системы безопасности не только корпоративных локальных сетей стандарта Ethernet, но также и беспроводных сетей стандарта Wi-Fi. 

Пакетные снифферы (устройства перехвата сетевых пакетов) могут быть использованы для анализа трафика в режиме реального времени либо же сравнения ожидаемого сетевого трафика и перехваченного. Данные устройства также могут быть использованы для сбора статистики, обнаружения сетвых атак или посторонних беспроводных устройств.

Пакетный сниффер на основе миникомпьютера Raspberry Pi и свободно распространяемого программного обеспечения может стать недорогой альтернативой подоброго рода устройств. 

В качестве задания на курсовую работу была поставлена задача разработать пассивный перехватчик пакетов на базе Raspberry Pi, Wi-Fi-адаптера и существующих программ для перехвата и анализа сетевого трафика.


\newpage
\ESKDstyle{plain}
\section{Используемые прораммные и аппаратные средства, обоснование выбора и их описание}
\setcounter{figure}{0}
\subsection{Raspberry Pi + TPLink}
%\input{technical_things/git}
\subsection{Ubuntu}
%\input{technical_things/git}
\subsection{Raspbian}
%\input{technical_things/git}
\subsection{tcpdump}
%\input{technical_things/git}
\subsection{wireshark}
%\input{technical_things/git}


\newpage
\ESKDstyle{plain}
\section{Проектирование и настройка пакетного сниффера}
\setcounter{figure}{0}
\subsection{Установка ОС Raspbian}
Для установки оперционной системмы Raspbian на Raspberry Pi необходимо перейти на страницу загрузок на официальном сайте~\cite{raspbian_downloads} и скачать необходимый образ ОС (рис.~\ref{raspbian_1:raspbian_1}), затем установить загруженный образ на SD-карту, которая впоследствии будет подключена к Raspberry Pi и с которой непосредственно будет загружаться система.

\begin{figure}[h!]
\center{\includegraphics[width=0.8\linewidth]{raspbian_1}}
\caption{ Официальная страница загрузок ОС Raspbian }
\label{raspbian_1:raspbian_1}
\end{figure}

Инструкцию по установке образа системы на SD-карту можно найти, перейдя по ссылке~\cite{raspbian_install}. 

После распаковки заргуженного архива с образом системы, необходимо вставить SD-карту в слот и выполнить следующие команды:

\begin{verbatim}
$ df -h   # увидеть все примонтированные устройства 
$ umount /dev/<ИМЯ_УСТРОЙСТВА>    # отмонтировать SD-карту
$ dd bs=4M if=2015-11-21-raspbian-jessie.img of=/dev/sdd    # записать образ
$ sync
\end{verbatim}

Далее достаточно извлечь SD-карту и установить ее в соответсвующий разъем Raspberry Pi.

\clearpage

\subsection{настройка сети (Ethernet)}
Для того, чтобы удаленно зайти в систему на Raspberry Pi, необходимо подключить к нему питание (система автоматически загрузится с SD-карты), подсоединиться по Ethernet-кабелю и осуществить необходимые настройки во вкладке <<Параметры системы>> --- <<Сеть>> --- <<Проводное>> --- <<Параметры>>. Установить параметры IPv4 и IPv6, как это продемонстрировано на рисунках~\ref{network_1:network_1} и~\ref{network_2:network_2} соответственно. Все остальные настройки оставить по умолчанию.

Результатом станет рабочее проводное соединение (рис.~\ref{network_3:network_3}). 

\begin{figure}[h!]
\center{\includegraphics[width=0.6\linewidth]{network_1}}
\caption{ Параметры IPv4 для проводного соединения }
\label{network_1:network_1}
\end{figure}

\begin{figure}[h!]
\center{\includegraphics[width=0.6\linewidth]{network_2}}
\caption{ Параметры IPv6 для проводного соединения }
\label{network_2:network_2}
\end{figure}

\begin{figure}[h!]
\center{\includegraphics[width=0.6\linewidth]{network_3}}
\caption{ Сведения о проводном соединении }
\label{network_3:network_3}
\end{figure}

\clearpage





\subsection{Установка ssh-соединения}
Для начала выведем список доступных сетевых интерфейсов из ARP-кэша при помощи следующей команды:

\begin{verbatim}
$ arp -vn
\end{verbatim}

Результат выполнения команды представлен на рисунке~\ref{ssh_1:ssh_1}.

Из полученной информации можно сделать вывод, что Raspberry Pi имеет IP-адрес в локальной сети 10.42.0.67. Для того, чтобы удаленно подсоединиться к нему по ssh, необходимо выполнить команду: 

\begin{verbatim}
$ ssh pi@10.42.0.67
\end{verbatim}

Далее система запрашивает пароль от Raspberry, после чего устанавливается ssh-соединение (рис.~\ref{ssh_2:ssh_2}).


\begin{figure}[h!]
\center{\includegraphics[width=0.8\linewidth]{ssh_1}}
\caption{ Результат выполнения команды arp -vn }
\label{ssh_1:ssh_1}
\end{figure}

\begin{figure}[h!]
\center{\includegraphics[width=0.8\linewidth]{ssh_2}}
\caption{ Установка ssh-соединения }
\label{ssh_2:ssh_2}
\end{figure}

\clearpage

\subsection{Установка драйвера TP-Link}
Необходимо узнать (проверить) версию системы (Raspbian) с помощью команды (рис.~\ref{driver_1:driver_1}):

\begin{verbatim}
$ uname -a
\end{verbatim} 
 
Далее проверить список доступных USB-устройств (рис.~\ref{driver_1:driver_1}). Запись <<ID 0BDA:8179>> и версия ОС будут определять, какой драйвер необходим. 

\begin{figure}[h!]
\center{\includegraphics[width=0.8\linewidth]{driver_1}}
\caption{ Текущая версия системы }
\label{driver_1:driver_1}
\end{figure}


\begin{figure}[h!]
\center{\includegraphics[width=0.5\linewidth]{driver_2}}
\caption{ Список USB-устройств }
\label{driver_2:driver_2}
\end{figure}

Подробную инструкцию можно найти, перейдя по ссылке \cite{rpi-tp}.

Остается загрузить необходимый драйвер (рис.~\ref{driver_3:driver_3}), установить его и перезагрузиться (рис.~\ref{driver_4:driver_4}). В случае успешной установки драйвера, команда ifconfig отобразит наличие беспроводного соединения wlan0 (рис.~\ref{driver_5:driver_5}).

\begin{figure}[h!]
\center{\includegraphics[width=0.8\linewidth]{driver_3}}
\caption{ Загрузка драйвера }
\label{driver_3:driver_3}
\end{figure}

\begin{figure}[h!]
\center{\includegraphics[width=0.6\linewidth]{driver_4}}
\caption{ Установка драйвера }
\label{driver_4:driver_4}
\end{figure}

\begin{figure}[h!]
\center{\includegraphics[width=0.6\linewidth]{driver_5}}
\caption{ Результат выполнения команды ifconfig }
\label{driver_5:driver_5}
\end{figure}

\clearpage




\subsection{Настройка режима monitor mode}
Для начала попробуем установить Wi-Fi соединение и подсоединиться к RPi без Ethernet-соединения. 

\begin{figure}[h!]
\center{\includegraphics[width=0.6\linewidth]{mm_1}}
\caption{ просканировали сеть }
\label{mm_1:mm_1}
\end{figure}


\begin{figure}[h!]
\center{\includegraphics[width=0.6\linewidth]{mm_2}}
\caption{ путь к config }
\label{mm_2:mm_2}
\end{figure}


\begin{figure}[h!]
\center{\includegraphics[width=0.3\linewidth]{mm_3}}
\caption{ config }
\label{mm_3:mm_3}
\end{figure}


\begin{figure}[h!]
\center{\includegraphics[width=0.7\linewidth]{mm_4}}
\caption{ config }
\label{mm_4:mm_4}
\end{figure}

подняли сетку в режиме Managed

\begin{figure}[h!]
\center{\includegraphics[width=0.7\linewidth]{mm_5}}
\caption{ config }
\label{mm_5:mm_5}
\end{figure}


тут можно рассказать про разные режимы даптеров (их 6) и про мониторящий режим. Устройство может в определенный момент времени находиться только в одном режиме.

Настроим адаптер в режим приема пакетов. Для этого необходимо ввести команду 
sudo iwconfig wlan0 mode monitor
При попытке ввести данную команду видим следующее сообщение:


\begin{figure}[h!]
\center{\includegraphics[width=0.6\linewidth]{mm_6}}
\caption{ ресурс занят }
\label{mm_6:mm_6}
\end{figure}

это связано с тем, что в системе работает network interface plugging daemon (ссылка на распберри пи орг)
необходимо его отключить для того, чтобы перевести устройство в другой режим работы

напишем небольшой bash-скрипт, который будет переводить адаптер в режим перехвата пакетов и содержать следующие команды:
sudo service ifplugd stop #останавливаем работу демона
sudo ifconfig wlan0 down #отключаем wi-fi соединение
sudo iwconfig wlan0 mode monitor #включаем прослушивающий режим
sudo ifconfig wlan0 up #включаем wi-fi соединение
sudo service ifplugd start #запускаем демона
iwconfig #проверям настройки

результат работы скрипта приведен на рисунке ... Устройство теперь в режиме перехвата пакетов. Данный скрипт необходимо запускать снова при перезапуске системы, поскольку по умолчанию устройство переходит в режим managed 

\begin{figure}[h!]
\center{\includegraphics[width=0.6\linewidth]{mm_7}}
\caption{ результат выполнения скрипта }
\label{mm_7:mm_7}
\end{figure}

\clearpage








\subsection{tcpdump}
%\input{technical_things/git}
\subsection{wireshark}
%\input{technical_things/git}



\newpage
\ESKDstyle{plain}
\setcounter{section}{0}
\section*{Заключение}
\addcontentsline{toc}{section}{Заключение}
Результатом проделанной работы стал перехватчик сетевых пакетов на основе недорогих аппаратных средств и существующих open-sourse-программ, пригодный для перехвата и дальнейшего анализа трафика. Сниффер может быть использован как в научно-образовательных целях, так и для обеспечения безопасности беспроводных сетей внутри какой-либо организации. 



\clearpage
\renewcommand{\refname}{Список использованных источников}
\addcontentsline{toc}{section}{Список использованных источников}
\bibliography{lit}

\end{document}
